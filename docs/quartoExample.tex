% Options for packages loaded elsewhere
% Options for packages loaded elsewhere
\PassOptionsToPackage{unicode}{hyperref}
\PassOptionsToPackage{hyphens}{url}
\PassOptionsToPackage{dvipsnames,svgnames,x11names}{xcolor}
%
\documentclass[
  11pt,
  a4paper,
  DIV=11,
  numbers=noendperiod]{scrartcl}
\usepackage{xcolor}
\usepackage{amsmath,amssymb}
\setcounter{secnumdepth}{5}
\usepackage{iftex}
\ifPDFTeX
  \usepackage[T1]{fontenc}
  \usepackage[utf8]{inputenc}
  \usepackage{textcomp} % provide euro and other symbols
\else % if luatex or xetex
  \usepackage{unicode-math} % this also loads fontspec
  \defaultfontfeatures{Scale=MatchLowercase}
  \defaultfontfeatures[\rmfamily]{Ligatures=TeX,Scale=1}
\fi
\usepackage[]{newpxtext}
\ifPDFTeX\else
  % xetex/luatex font selection
\fi
% Use upquote if available, for straight quotes in verbatim environments
\IfFileExists{upquote.sty}{\usepackage{upquote}}{}
\IfFileExists{microtype.sty}{% use microtype if available
  \usepackage[]{microtype}
  \UseMicrotypeSet[protrusion]{basicmath} % disable protrusion for tt fonts
}{}
% Make \paragraph and \subparagraph free-standing
\makeatletter
\ifx\paragraph\undefined\else
  \let\oldparagraph\paragraph
  \renewcommand{\paragraph}{
    \@ifstar
      \xxxParagraphStar
      \xxxParagraphNoStar
  }
  \newcommand{\xxxParagraphStar}[1]{\oldparagraph*{#1}\mbox{}}
  \newcommand{\xxxParagraphNoStar}[1]{\oldparagraph{#1}\mbox{}}
\fi
\ifx\subparagraph\undefined\else
  \let\oldsubparagraph\subparagraph
  \renewcommand{\subparagraph}{
    \@ifstar
      \xxxSubParagraphStar
      \xxxSubParagraphNoStar
  }
  \newcommand{\xxxSubParagraphStar}[1]{\oldsubparagraph*{#1}\mbox{}}
  \newcommand{\xxxSubParagraphNoStar}[1]{\oldsubparagraph{#1}\mbox{}}
\fi
\makeatother


\usepackage{longtable,booktabs,array}
\usepackage{calc} % for calculating minipage widths
% Correct order of tables after \paragraph or \subparagraph
\usepackage{etoolbox}
\makeatletter
\patchcmd\longtable{\par}{\if@noskipsec\mbox{}\fi\par}{}{}
\makeatother
% Allow footnotes in longtable head/foot
\IfFileExists{footnotehyper.sty}{\usepackage{footnotehyper}}{\usepackage{footnote}}
\makesavenoteenv{longtable}
\usepackage{graphicx}
\makeatletter
\newsavebox\pandoc@box
\newcommand*\pandocbounded[1]{% scales image to fit in text height/width
  \sbox\pandoc@box{#1}%
  \Gscale@div\@tempa{\textheight}{\dimexpr\ht\pandoc@box+\dp\pandoc@box\relax}%
  \Gscale@div\@tempb{\linewidth}{\wd\pandoc@box}%
  \ifdim\@tempb\p@<\@tempa\p@\let\@tempa\@tempb\fi% select the smaller of both
  \ifdim\@tempa\p@<\p@\scalebox{\@tempa}{\usebox\pandoc@box}%
  \else\usebox{\pandoc@box}%
  \fi%
}
% Set default figure placement to htbp
\def\fps@figure{htbp}
\makeatother


% definitions for citeproc citations
\NewDocumentCommand\citeproctext{}{}
\NewDocumentCommand\citeproc{mm}{%
  \begingroup\def\citeproctext{#2}\cite{#1}\endgroup}
\makeatletter
 % allow citations to break across lines
 \let\@cite@ofmt\@firstofone
 % avoid brackets around text for \cite:
 \def\@biblabel#1{}
 \def\@cite#1#2{{#1\if@tempswa , #2\fi}}
\makeatother
\newlength{\cslhangindent}
\setlength{\cslhangindent}{1.5em}
\newlength{\csllabelwidth}
\setlength{\csllabelwidth}{3em}
\newenvironment{CSLReferences}[2] % #1 hanging-indent, #2 entry-spacing
 {\begin{list}{}{%
  \setlength{\itemindent}{0pt}
  \setlength{\leftmargin}{0pt}
  \setlength{\parsep}{0pt}
  % turn on hanging indent if param 1 is 1
  \ifodd #1
   \setlength{\leftmargin}{\cslhangindent}
   \setlength{\itemindent}{-1\cslhangindent}
  \fi
  % set entry spacing
  \setlength{\itemsep}{#2\baselineskip}}}
 {\end{list}}
\usepackage{calc}
\newcommand{\CSLBlock}[1]{\hfill\break\parbox[t]{\linewidth}{\strut\ignorespaces#1\strut}}
\newcommand{\CSLLeftMargin}[1]{\parbox[t]{\csllabelwidth}{\strut#1\strut}}
\newcommand{\CSLRightInline}[1]{\parbox[t]{\linewidth - \csllabelwidth}{\strut#1\strut}}
\newcommand{\CSLIndent}[1]{\hspace{\cslhangindent}#1}



\setlength{\emergencystretch}{3em} % prevent overfull lines

\providecommand{\tightlist}{%
  \setlength{\itemsep}{0pt}\setlength{\parskip}{0pt}}



 


\usepackage{setspace}
\usepackage{float}
\usepackage{booktabs}
\usepackage{makecell}
\usepackage{threeparttable}
\usepackage{tabularx}
\usepackage{multirow}
\usepackage{array}
\KOMAoption{captions}{tableheading}
\makeatletter
\@ifpackageloaded{caption}{}{\usepackage{caption}}
\AtBeginDocument{%
\ifdefined\contentsname
  \renewcommand*\contentsname{Table of contents}
\else
  \newcommand\contentsname{Table of contents}
\fi
\ifdefined\listfigurename
  \renewcommand*\listfigurename{List of Figures}
\else
  \newcommand\listfigurename{List of Figures}
\fi
\ifdefined\listtablename
  \renewcommand*\listtablename{List of Tables}
\else
  \newcommand\listtablename{List of Tables}
\fi
\ifdefined\figurename
  \renewcommand*\figurename{Figure}
\else
  \newcommand\figurename{Figure}
\fi
\ifdefined\tablename
  \renewcommand*\tablename{Table}
\else
  \newcommand\tablename{Table}
\fi
}
\@ifpackageloaded{float}{}{\usepackage{float}}
\floatstyle{ruled}
\@ifundefined{c@chapter}{\newfloat{codelisting}{h}{lop}}{\newfloat{codelisting}{h}{lop}[chapter]}
\floatname{codelisting}{Listing}
\newcommand*\listoflistings{\listof{codelisting}{List of Listings}}
\makeatother
\makeatletter
\makeatother
\makeatletter
\@ifpackageloaded{caption}{}{\usepackage{caption}}
\@ifpackageloaded{subcaption}{}{\usepackage{subcaption}}
\makeatother
\usepackage{bookmark}
\IfFileExists{xurl.sty}{\usepackage{xurl}}{} % add URL line breaks if available
\urlstyle{same}
\hypersetup{
  pdftitle={Quarto and MakeTables},
  pdfauthor={Peter Pan},
  colorlinks=true,
  linkcolor={black},
  filecolor={Maroon},
  citecolor={black},
  urlcolor={Blue},
  pdfcreator={LaTeX via pandoc}}


\title{Quarto and MakeTables}
\author{Peter Pan\footnote{University of Nowhere}}
\date{May 1, 2025}
\begin{document}
\maketitle
\begin{abstract}
This document illustrates how to use MakeTables to write academic
papers. Just write your paper in markdown in Jupyter Notebooks. Tables
are displayed as HTML interactively when you work in the notebook on the
screen. When you render the notebook as PDF with Quarto, MakeTable
generates LaTeX code used by Quarto.

\newpage
\end{abstract}


\onehalfspacing

\section{Introduction}\label{introduction}

A very convenient way to write academic papers which include the code
for all reported analyses is to use Jupyter notebooks together with
\href{https://quarto.org/}{Quarto}. Quarto is a scientific publishing
system that allows you to combine text, code, and results in a single
document. Quarto will render Jupyter notebooks to PDF using LaTeX. When
you include tables created with MakeTables in your notebook, they will
be rendered directly on the screen in the notebook (as HTML via
\href{https://posit-dev.github.io/great-tables/articles/intro.html}{Great
Tables}) but will be rendered as LaTeX tables in the final PDF document
by Quarto.

Here is simple example of how to use MakeTables in a notebook to write a
paper. It generates tables, for instance based on regressions in either
\href{https://py-econometrics.github.io/pyfixest/pyfixest.html}{pyfixest}
or \href{https://www.statsmodels.org/stable/index.html}{statsmodels}.
Just render this notebook with
\texttt{quarto\ render\ quartoExample.ipynb} to create the PDF document.

Note it is also very easy to include references and citations in your
Quarto document (see for instance Wooldridge
(\citeproc{ref-wooldridge2010econometric}{2010}) and Greene
(\citeproc{ref-greene2018econometric}{2018})).

\section{Some Tables}\label{some-tables}

Table Table~\ref{tbl-1} shows some descriptive statistics.

\begin{table}[H]

\caption{\label{tbl-1}Automotive Data Summary}

\centering{

\renewcommand\cellalign{t}
\begin{threeparttable}
\begingroup
\renewcommand\arraystretch{1}
\setlength{\tabcolsep}{3pt}
\begin{tabularx}{\linewidth}{@{}p{3cm}>{\centering\arraybackslash}X>{\centering\arraybackslash}X>{\centering\arraybackslash}X>{\centering\arraybackslash}X>{\centering\arraybackslash}X>{\centering\arraybackslash}X}
\toprule
 & \multicolumn{3}{c}{Domestic} & \multicolumn{3}{c}{Foreign} \\
\cmidrule(lr){2-4} \cmidrule(lr){5-7}
 & N & Mean & Std. Dev. & N & Mean & Std. Dev. \\
\midrule
\addlinespace[1ex]
Mileage (mpg) & 52 & 19.83 & 4.74 & 22 & 24.77 & 6.61 \\
\addlinespace[0.5ex]
\addlinespace[0.5ex]
Weight (lbs.) & 52 & 3317.12 & 695.36 & 22 & 2315.91 & 433.00 \\
\addlinespace[0.5ex]
\addlinespace[0.5ex]
Price & 52 & 6072.42 & 3097.10 & 22 & 6384.68 & 2621.92 \\
\addlinespace[0.5ex]
\bottomrule
\end{tabularx}
\endgroup
\footnotesize 
\noindent\begin{minipage}{\linewidth}\smallskip\footnotesize
\end{minipage}

\end{threeparttable}

\begin{verbatim}
\end{verbatim}

}

\end{table}%

Table Table~\ref{tbl-2} shows regression results here estimated with the
excellent \texttt{pyfixest} package.

\begin{table}[H]

\caption{\label{tbl-2}A Regression Table (PyFixest)}

\centering{

\renewcommand\cellalign{t}
\begin{threeparttable}
\begingroup
\renewcommand\arraystretch{1}
\setlength{\tabcolsep}{3pt}
\begin{tabularx}{\linewidth}{@{}>{\raggedright\arraybackslash}X>{\centering\arraybackslash}X>{\centering\arraybackslash}X}
\toprule
 & \multicolumn{2}{c}{Mileage (mpg)} \\
\cmidrule(lr){2-3}
 & (1) & (2) \\
\midrule
\addlinespace[1ex]
Weight (lbs.) & \makecell{-0.006*** \\ (0.001)} & \makecell{-0.004* \\ (0.002)} \\
\addlinespace[0.5ex]
\addlinespace[0.5ex]
Length (in.) &  & \makecell{-0.080 \\ (0.055)} \\
\addlinespace[0.5ex]
\addlinespace[0.5ex]
Intercept & \makecell{39.440*** \\ (1.614)} & \makecell{47.885*** \\ (6.088)} \\
\addlinespace[0.5ex]
\midrule
\addlinespace[1ex]
Observations & 74 & 74 \\
\addlinespace[0.5ex]
\addlinespace[0.5ex]
$R^2$ & 0.652 & 0.661 \\
\addlinespace[0.5ex]
\bottomrule
\end{tabularx}
\endgroup
\footnotesize 
\noindent\begin{minipage}{\linewidth}\smallskip\footnotesize
Significance levels: * p < 0.05, ** p < 0.01, *** p < 0.001. Format of coefficient cell: Coefficient   (Std. Error)\end{minipage}

\end{threeparttable}

\begin{verbatim}
\end{verbatim}

}

\end{table}%

Table \textbf{?@tbl-3} uses \texttt{statsmodels} instead to estimate a
probit model.

\renewcommand\cellalign{t}
\begin{threeparttable}
\begingroup
\renewcommand\arraystretch{1}
\setlength{\tabcolsep}{3pt}
\begin{tabularx}{\linewidth}{@{}>{\raggedright\arraybackslash}X>{\centering\arraybackslash}X>{\centering\arraybackslash}X}
\toprule
 & \multicolumn{2}{c}{Foreign (indicator)} \\
\cmidrule(lr){2-3}
 & \multicolumn{1}{c}{OLS} & \multicolumn{1}{c}{Probit} \\
\cmidrule(lr){2-2} \cmidrule(lr){3-3}
 & (1) & (2) \\
\midrule
\addlinespace[1ex]
Weight (lbs.) & \makecell{-0.001*** \\ (0.000)} & \makecell{-0.004** \\ (0.002)} \\
\addlinespace[0.5ex]
\addlinespace[0.5ex]
Length (in.) & \makecell{0.007 \\ (0.005)} & \makecell{0.033 \\ (0.045)} \\
\addlinespace[0.5ex]
\addlinespace[0.5ex]
Price & \makecell{0.000*** \\ (0.000)} & \makecell{0.001*** \\ (0.000)} \\
\addlinespace[0.5ex]
\addlinespace[0.5ex]
Intercept & \makecell{0.656 \\ (0.583)} & \makecell{1.327 \\ (4.992)} \\
\addlinespace[0.5ex]
\midrule
\addlinespace[1ex]
Observations & 74 & 74 \\
\addlinespace[0.5ex]
\addlinespace[0.5ex]
$R^2$ & 0.552 & - \\
\addlinespace[0.5ex]
\addlinespace[0.5ex]
Pseudo $R^2$ & - & 0.606 \\
\addlinespace[0.5ex]
\bottomrule
\end{tabularx}
\endgroup
\footnotesize 
\noindent\begin{minipage}{\linewidth}\smallskip\footnotesize
Significance levels: * p < 0.05, ** p < 0.01, *** p < 0.001. Format of coefficient cell: Coefficient   (Std. Error)\end{minipage}

\end{threeparttable}

\begin{verbatim}
\end{verbatim}

\section{Conclusion}\label{conclusion}

Lorem ipsum dolor sit amet, consectetur adipiscing elit. Sed do eiusmod
tempor incididunt ut labore et dolore magna aliqua. Ut enim ad minim
veniam, quis nostrud exercitation ullamco laboris nisi ut aliquip ex ea
commodo consequat. Duis aute irure dolor in reprehenderit in voluptate
velit esse cillum dolore eu fugiat nulla pariatur. Excepteur sint
occaecat cupidatat non proident, sunt in culpa qui officia deserunt
mollit anim id est laborum.

\newpage{}

\section{References}\label{references}

\phantomsection\label{refs}
\begin{CSLReferences}{1}{0}
\bibitem[\citeproctext]{ref-greene2018econometric}
Greene, W. H. (2018). \emph{Econometric analysis} (8th ed.). Pearson.

\bibitem[\citeproctext]{ref-wooldridge2010econometric}
Wooldridge, J. M. (2010). \emph{Econometric analysis of cross section
and panel data}.

\end{CSLReferences}




\end{document}
