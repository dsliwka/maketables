% Options for packages loaded elsewhere
% Options for packages loaded elsewhere
\PassOptionsToPackage{unicode}{hyperref}
\PassOptionsToPackage{hyphens}{url}
\PassOptionsToPackage{dvipsnames,svgnames,x11names}{xcolor}
%
\documentclass[
  11pt,
  a4paper,
  DIV=11,
  numbers=noendperiod]{scrartcl}
\usepackage{xcolor}
\usepackage{amsmath,amssymb}
\setcounter{secnumdepth}{5}
\usepackage{iftex}
\ifPDFTeX
  \usepackage[T1]{fontenc}
  \usepackage[utf8]{inputenc}
  \usepackage{textcomp} % provide euro and other symbols
\else % if luatex or xetex
  \usepackage{unicode-math} % this also loads fontspec
  \defaultfontfeatures{Scale=MatchLowercase}
  \defaultfontfeatures[\rmfamily]{Ligatures=TeX,Scale=1}
\fi
\usepackage[]{newpxtext}
\ifPDFTeX\else
  % xetex/luatex font selection
\fi
% Use upquote if available, for straight quotes in verbatim environments
\IfFileExists{upquote.sty}{\usepackage{upquote}}{}
\IfFileExists{microtype.sty}{% use microtype if available
  \usepackage[]{microtype}
  \UseMicrotypeSet[protrusion]{basicmath} % disable protrusion for tt fonts
}{}
% Make \paragraph and \subparagraph free-standing
\makeatletter
\ifx\paragraph\undefined\else
  \let\oldparagraph\paragraph
  \renewcommand{\paragraph}{
    \@ifstar
      \xxxParagraphStar
      \xxxParagraphNoStar
  }
  \newcommand{\xxxParagraphStar}[1]{\oldparagraph*{#1}\mbox{}}
  \newcommand{\xxxParagraphNoStar}[1]{\oldparagraph{#1}\mbox{}}
\fi
\ifx\subparagraph\undefined\else
  \let\oldsubparagraph\subparagraph
  \renewcommand{\subparagraph}{
    \@ifstar
      \xxxSubParagraphStar
      \xxxSubParagraphNoStar
  }
  \newcommand{\xxxSubParagraphStar}[1]{\oldsubparagraph*{#1}\mbox{}}
  \newcommand{\xxxSubParagraphNoStar}[1]{\oldsubparagraph{#1}\mbox{}}
\fi
\makeatother


\usepackage{longtable,booktabs,array}
\usepackage{calc} % for calculating minipage widths
% Correct order of tables after \paragraph or \subparagraph
\usepackage{etoolbox}
\makeatletter
\patchcmd\longtable{\par}{\if@noskipsec\mbox{}\fi\par}{}{}
\makeatother
% Allow footnotes in longtable head/foot
\IfFileExists{footnotehyper.sty}{\usepackage{footnotehyper}}{\usepackage{footnote}}
\makesavenoteenv{longtable}
\usepackage{graphicx}
\makeatletter
\newsavebox\pandoc@box
\newcommand*\pandocbounded[1]{% scales image to fit in text height/width
  \sbox\pandoc@box{#1}%
  \Gscale@div\@tempa{\textheight}{\dimexpr\ht\pandoc@box+\dp\pandoc@box\relax}%
  \Gscale@div\@tempb{\linewidth}{\wd\pandoc@box}%
  \ifdim\@tempb\p@<\@tempa\p@\let\@tempa\@tempb\fi% select the smaller of both
  \ifdim\@tempa\p@<\p@\scalebox{\@tempa}{\usebox\pandoc@box}%
  \else\usebox{\pandoc@box}%
  \fi%
}
% Set default figure placement to htbp
\def\fps@figure{htbp}
\makeatother





\setlength{\emergencystretch}{3em} % prevent overfull lines

\providecommand{\tightlist}{%
  \setlength{\itemsep}{0pt}\setlength{\parskip}{0pt}}



 


\usepackage{setspace}
\usepackage{float}
\usepackage{booktabs}
\usepackage{makecell}
\usepackage{threeparttable}
\usepackage{tabularx}
\usepackage{multirow}
\usepackage{array}
\KOMAoption{captions}{tableheading}
\makeatletter
\@ifpackageloaded{caption}{}{\usepackage{caption}}
\AtBeginDocument{%
\ifdefined\contentsname
  \renewcommand*\contentsname{Table of contents}
\else
  \newcommand\contentsname{Table of contents}
\fi
\ifdefined\listfigurename
  \renewcommand*\listfigurename{List of Figures}
\else
  \newcommand\listfigurename{List of Figures}
\fi
\ifdefined\listtablename
  \renewcommand*\listtablename{List of Tables}
\else
  \newcommand\listtablename{List of Tables}
\fi
\ifdefined\figurename
  \renewcommand*\figurename{Figure}
\else
  \newcommand\figurename{Figure}
\fi
\ifdefined\tablename
  \renewcommand*\tablename{Table}
\else
  \newcommand\tablename{Table}
\fi
}
\@ifpackageloaded{float}{}{\usepackage{float}}
\floatstyle{ruled}
\@ifundefined{c@chapter}{\newfloat{codelisting}{h}{lop}}{\newfloat{codelisting}{h}{lop}[chapter]}
\floatname{codelisting}{Listing}
\newcommand*\listoflistings{\listof{codelisting}{List of Listings}}
\makeatother
\makeatletter
\makeatother
\makeatletter
\@ifpackageloaded{caption}{}{\usepackage{caption}}
\@ifpackageloaded{subcaption}{}{\usepackage{subcaption}}
\makeatother
\usepackage{bookmark}
\IfFileExists{xurl.sty}{\usepackage{xurl}}{} % add URL line breaks if available
\urlstyle{same}
\hypersetup{
  pdftitle={This is a quarto test},
  pdfauthor={Peter Pan},
  colorlinks=true,
  linkcolor={black},
  filecolor={Maroon},
  citecolor={black},
  urlcolor={Blue},
  pdfcreator={LaTeX via pandoc}}


\title{This is a quarto test}
\author{Peter Pan\footnote{University of Nowhere}}
\date{May 1, 2025}
\begin{document}
\maketitle
\begin{abstract}
This document illustrates how to use TabOut in to write papers in
Jupyter Notebooks such that tables are either shown as html on the
screen or as latex code when rendered to pdf with quarto. \newpage
\end{abstract}


\onehalfspacing

\section{Introduction}\label{introduction}

Lore ipsum dolor sit amet, consectetur adipiscing elit. Sed do eiusmod
tempor incididunt ut labore et dolore magna aliqua. Ut enim ad minim
veniam, quis nostrud exercitation ullamco laboris nisi ut aliquip ex ea
commodo consequat. Duis aute irure dolor in reprehenderit in voluptate
velit esse cillum dolore eu fugiat nulla pariatur. Excepteur sint
occaecat cupidatat non proident, sunt in culpa qui officia deserunt
mollit anim id est laborum.

\begin{table}[H]

\caption{\label{tbl-1}The First Table}

\centering{

\renewcommand\cellalign{t}
\begin{threeparttable}
\begin{tabularx}{\linewidth}{>{\raggedright\arraybackslash}X>{\centering\arraybackslash}X>{\centering\arraybackslash}X>{\centering\arraybackslash}X>{\centering\arraybackslash}X}
\toprule
 & \multicolumn{2}{c}{France} & \multicolumn{2}{c}{US} \\
\cmidrule(lr){2-3} \cmidrule(lr){4-5}
 & High & Low & High & Low \\
\midrule
\addlinespace
\emph{Gr 1} \\
\addlinespace
Var 1 & -0.485 & 1.217 & -0.292 & 0.492 \\
Var 2 & 0.376 & 0.605 & 0.337 & 0.132 \\
Var 3 & 0.094 & -0.262 & 1.416 & 0.869 \\
\addlinespace
\midrule
\addlinespace
\emph{Gr 2} \\
\addlinespace
Var 4 & 0.394 & -0.245 & -0.138 & 0.67 \\
Var 5 & -1.515 & -2.077 & -0.237 & 0.907 \\
\addlinespace
\midrule
\addlinespace
\emph{Gr 3} \\
\addlinespace
Var 6 & 0.515 & -0.075 & 0.404 & -1.03 \\
\bottomrule
\end{tabularx}
\footnotesize 
\noindent\begin{minipage}{\linewidth}\smallskip\footnotesize
Sample notes!\end{minipage}

\end{threeparttable}

\begin{verbatim}
\end{verbatim}

}

\end{table}%

\begin{table}[H]

\caption{\label{tbl-2}The Second Table}

\centering{

\renewcommand\cellalign{t}
\begin{threeparttable}
\begin{tabularx}{\linewidth}{>{\raggedright\arraybackslash}X>{\centering\arraybackslash}X>{\centering\arraybackslash}X>{\centering\arraybackslash}X}
\toprule
 & Control & Treatment 1 & Treatment 2 \\
\midrule
\addlinespace
Ability (test score) & \makecell{99.84\\(16.33)} & \makecell{100.32\\(14.95)} & \makecell{99.37\\(15.63)} \\
Age (years) & \makecell{38.46\\(12.85)} & \makecell{36.47\\(11.47)} & \makecell{38.65\\(12.54)} \\
\addlinespace
\midrule
\addlinespace
N & 358 & 298 & 344 \\
\bottomrule
\end{tabularx}
\footnotesize 
\noindent\begin{minipage}{\linewidth}\smallskip\footnotesize
Note: Displayed statistics are Mean (Std. Dev.).\end{minipage}

\end{threeparttable}

\begin{verbatim}
\end{verbatim}

}

\end{table}%

\begin{table}[H]

\caption{\label{tbl-3}The Second Table}

\centering{

\renewcommand\cellalign{t}
\begin{threeparttable}
\begin{tabularx}{\linewidth}{>{\raggedright\arraybackslash}X>{\centering\arraybackslash}X>{\centering\arraybackslash}X}
\toprule
 & \multicolumn{2}{c}{Sales (Euro)} \\
\cmidrule(lr){2-3}
 & (1) & (2) \\
\midrule
\addlinespace
Ability (test score) & \makecell{99.748*** \\ (13.088)} & \makecell{99.947*** \\ (10.552)} \\
Age (years) & \makecell{14.105 \\ (16.586)} & \makecell{10.528 \\ (13.374)} \\
training &  & \makecell{8209.396*** \\ (354.001)} \\
Intercept & \makecell{2.064E+04*** \\ (1446.138)} & \makecell{1.812E+04*** \\ (1170.985)} \\
\addlinespace
\midrule
\addlinespace
Observations & 1000 & 1000 \\
R² & 0.056 & 0.387 \\
\bottomrule
\end{tabularx}
\footnotesize 
\noindent\begin{minipage}{\linewidth}\smallskip\footnotesize
Significance levels: * p < 0.05, ** p < 0.01, *** p < 0.001. Format of coefficient cell: Coefficient   (Std. Error)\end{minipage}

\end{threeparttable}

\begin{verbatim}
\end{verbatim}

}

\end{table}%

\section{Conclusion}\label{conclusion}

Lore ipsum dolor sit amet, consectetur adipiscing elit. Sed do eiusmod
tempor incididunt ut labore et dolore magna aliqua. Ut enim ad minim
veniam, quis nostrud exercitation ullamco laboris nisi ut aliquip ex ea
commodo consequat. Duis aute irure dolor in reprehenderit in voluptate
velit esse cillum dolore eu fugiat nulla pariatur. Excepteur sint
occaecat cupidatat non proident, sunt in culpa qui officia deserunt
mollit anim id est laborum.

\newpage{}

\section{References}\label{references}

\phantomsection\label{refs}




\end{document}
